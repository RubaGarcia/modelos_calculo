%%***************************************************
\documentclass[a4paper,12pt]{article}

%% PACKAGES
\usepackage{amssymb}% to have some extra types of fonts
%% - writing in Spanish
\usepackage[utf8]{inputenc}% To type Spanish accents
\usepackage[spanish]{babel}% to have Spanish captions
%% - doc. formatting
\usepackage[a4paper,left=3.1cm,right=3.1cm,bottom=2.5cm,top=2.5cm]{geometry}
%% - headings
\usepackage{fancyhdr}
%% - last page
\usepackage{lastpage}
%% - Hyperref
\usepackage{hyperref}

\usepackage{amsmath}


%% Headings and footing code
\fancyhead[L]{Universidad de Cantabria}
\fancyhead[C]{Grado en Matemáticas}
\fancyhead[R]{Grado en Ing. Informática}
\fancyfoot[R]{\thepage/\pageref{LastPage}}
\fancyfoot[C]{}


\title{\large{\textsc{Modelos de cálculo}} \\ \vspace{-2mm} \Large{Práctica 1}}

\author{Rubén García \and Javier Mier \and Yuhua Zhan}
\date{}

\pagestyle{fancy}

\begin{document}

\maketitle

\thispagestyle{fancy}

El objetivo de esta práctica es demostrar de manera formal que la función

$f: \mathbb{N} ^{2}\rightarrow \mathbb {N}$ definida por

$f(n,m)=\left\{
\begin{array}{cc}
    1&\text{si }resto(n,m) = 0\\
    0&\text{si }resto(n,m) \neq 0\\
\end{array}
\right.$

es primitiva recursiva (PR), sabiendo que:

Las funciones iniciales (a)-(c) son $primitivas recursivas$
\begin{itemize}
\item[(a)] La \textit{función cero} $z(n)=0, \forall n \in \mathbb {N} $
\item [(b)] La \textit{función sucesor} $s(n)=n+1, \forall n \in \mathbb {N} $
\item[(c)] La \textit{función proyección}:

$p^{k}_{i}(n_{1},...,n_{k})=n_{i},\forall k \in \mathbb {N}^{*},n_{1},...,n_{k} \in \mathbb{N}, \forall k \in {1,...,k}$

Dadas $g_1,g_2,h,h_0,...,h_l$
funciones primitivas recursivas, también lo son $f_1 :
\mathbb{N}^{k} \rightarrow \mathbb{N} y f_2: \mathbb{N}^{k+1} \rightarrow \mathbb{N}$ obtenidas a través

\item[(d)] de la \textit{composición}, $f_1(\vec{n}) = g_1(h_0(\vec{n}),...,h_l(\vec{n})),$ o

\item[(e)] de la \textit{recursión primitiva}: 

$f_2(\vec{n},0) = g_2(\vec{n})$,

$f_2(\vec{n},m+1) = h(\vec{n},m,f_2(\vec{n},m))$, 

donde $\vec{n} = (n_{1},...,n_{k})$.

La función $resto$ se define de la siguiente forma:

$resto(n,m)=
\begin{cases}
    \text{el resto de dividir $m$ por $n$}&\text{si }n \neq 0,\\
    m&\text{si }n = 0\\
\end{cases}
$

Se deben entregar los ficheros \texttt{practica1.tex} y \texttt{practica1.pdf} (la demostración se debe escribit en latex).

\end{itemize}

\newpage
Se observa que $f(n,m) = \overline{sg}(resto(n,m))$, donde


$\overline{sg}(n):=
\begin{cases}
1, & \text{ si } n = 0 \\
0, & \text{ si } n \neq 0
\end{cases}
$


Por lo tanto, si probásemos $\overline{sg}$ y \textit{resto} son primitivas recursivas(PR), podríamos concluir que $f$ también lo es (cf.\textit{(d)}).

Empezamos con las funciones básicas $suma,mult,pred,resta,abs,sg,$ definidas a continuación
\begin{itemize}
    \item $suma(n, m):=n+m$
    
    \item $mult(n, m):=n * m$
    
    \item $pred(n):= 
    \begin{cases}n-1, & \text { si } n \geq m \\
    0, & \text { si } n<m
    \end{cases}$
    
    \item $resta(n, m):= 
    \begin{cases}n-m, & \text { si } n \geq m \\
    0, & \text { si } n<m
    \end{cases}$
    
    \item $abs(n, m):= 
    \begin{cases}n-m, & \text { si } n \geq m \\
    m-n, & \text { si } n<m
    \end{cases}$
    
    \item $s g(n):= 
    \begin{cases}0, & \text { si } n=0 \\
    1, & \text { si } n \neq 0
    \end{cases}$
    
\end{itemize}

Vamos a mostrar que todas estas funciones son PR:

$\begin{cases}suma(n,0)=n=p^{1}_{1}(n) \text{ es PR por \textit{(c)}} \\
suma(n,m + 1) = h^{1}(n,m,suma(n,m)),
\end{cases}$

donde $h^{1}(x,y,z):=s(p_{3}^{3}(x,y,z))$ es una función PR por \textit{(b)},\textit{(c)} y \textit{(d)}

Por lo tanto, suma es una función PR por \textit{(e)}.

$\begin{cases}mult(n,0)=0=z(n) \text{ es PR por \textit{(a)}} \\
mult(n,m + 1) = h^{2}(n,m,mult(n,m)),
\end{cases}$

donde $h^{2}(x,y,z):=suma(p^{3}_{1}(x,y,z), p_{3}^{3}(x,y,z))$ es una función PR por  \textit{(c)},\textit{(d)} y porque \textit{suma} es PR

Por lo tanto, \textit{mult} es una función PR por \textit{(e)}.

$\begin{cases}pred(0)=0 \text{ es PR por ser constante} \\
pred(m + 1) = h^{3}(m,pred(m)),
\end{cases}$

donde $h^{3}(x,y):= p_{1}^{2}(x,y)$ es una función PR por \textit{(c)}.

Por lo tanto, \textit{pred} es una función PR por \textit{(e)}.

$\begin{cases}resta(n,0)=n=p^{1}_{1}(n) \text{ es PR por \textit{(c)}} \\
resta(n,m + 1) = h^{4}(n,m,resta(n,m)),
\end{cases}$

donde $h^{4}.=pred(p_{3}^{3}(x,y,z)))$ es una función PR por \textit{(c)},\textit{(d)} y porque \textit{pred} es PR.

Por lo tanto, \textit{resta} es una función PR por \textit{(e)}.

$abs(n,m)=suma(resta(n,m),h^{5}(n,m))$

donde $h^{5}(x,y):=resta(p_{2}^{2}(x,y),p_{1}^{2}(x,y))$ es una función PR por \textit{(c)},\textit{(d)} y porque \textit{resta} es PR

Por lo tanto, \textit{abs} es una funcion PR por \textit{(d)}

$\begin{cases}sg(0)=0 \text{ es PR por ser constante} \\
sg(m + 1) = h^{6}(m,sg(m)),
\end{cases}$

donde $h^{6}(x,y) := 1 =s(z(p_{1}^{2}(x,y)))$ es una función PR por \textit{(a)},\textit{(b)},\textit{(c)} y \textit{(d)}.

Por lo tanto, \textit{sg} es una función PR por \textit{(e)}.


Falta por demostrar que las funciones $\overline{sg}$ y $resto$ también son PR.

$\overline{sg}(n) = resta(s(z(n)), sg(n))$, y, por lo tanto, $\overline{sg}$ es una función PR por \textit{(a),(b),(d)} y porque $resta$ y $sg$ son funciones PR.

    $\begin{cases}
    resto(n,0) = 0 = z(n) \mbox{es PR por \textit{(a)}}\\
    resto(n,m+1) = h(n,m,resto(n,m)),
    \end{cases}$
    
donde $h(x,y,z) = mult(h^{7}(x,y,z),h^{8}(x,y,z)), h^{7}(x,y,z) = s(p^{3}_{3}(x,y,z)), h^{8}(x,y,z) = sg(abs(p^{3}_{1}(x,y,z),s(p^{3}_{3}(x,y,z))))$ son funciones PR:

\begin{itemize}
    \item $h^{7}$ es una función PR por \textit{(b), (c)} y \textit{(d)}
    \item $h^{8}$ en una función PR por \textit{(b), (c), (d)} y porque $sg$ y $abs$ son funciones PR.
    \item $h$ es una función PR por \textit{(d)} y porque $h^{7}, h^{8}$ y $mult$ son funciones PR.
\end{itemize}

Por lo tanto, $resto$ es una función PR por \textit{(e)}.

\end{document}
